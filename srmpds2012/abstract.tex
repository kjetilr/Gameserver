\begin{abstract}
Supporting thousands of interacting players in a virtual world poses
huge challenges with respect to processing. Existing work that
addresses the challenge utilizes a variety of spatial partitioning
algorithms to distribute the load.  If, however, a large number of
players needs to interact tightly across an area of the game world,
spatial partitioning cannot subdivide this area without incurring
massive communication costs, latency or inconsistency.  It is a major
challenge of game engines to scale such areas to the largest number of
players possible; in a deviation from earlier thinking, parallelism on
multi-core architectures is applied to increase scalability.  In this
paper, we evaluate the design and implementation of our game server
architecture, called LEARS, which allows for lock-free parallel
processing of a single spatial partition by considering every game
cycle an atomic tick.
%We also discuss challenges that arise as different
%dependency requirements are introduced to the system. 
Our prototype is evaluated using traces from live game sessions where
we measure the server response time for all objects that need timely
updates. We also measure how the response time for the multi-threaded
implementation varies with the number of threads used. Our results
show that the challenge of scaling up a game-server can be an
embarrassingly parallel problem.
\end{abstract}
