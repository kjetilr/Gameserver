\section{Discussion}
\label{sec:disc}

%\subsection {Spatial partitioning} \label{sec:spatialpartitioning}

Most approaches to multi-threaded game server implementations in the
literature (e.g., ~\cite{Abdelkhalek2004++}) use some form of
\textit{spatial partitioning} to lock parts of the game world while
allowing separate parts to run in parallel. Spatial partitioning is
also used in other situations to limit workload. The number of players
that game designers can allow in one area in a game server is limited
by the worst-case scenario. The worst case scenario for a spatially
partitioned game world is when everybody move to the same point, where
the spatial partitioning still ends up with everybody in the same
partition regardless of granularity. This paper investigates an
orthogonal and complementary approach which tries to increase the
maximum number of users in the worst case scenario where all players
can see each other at all times. Thus, spatial partitioning could be
added to further scale the game server.

%, as we want to investigate the worst case scenario, hence
%the numbers presented in this paper assume all players can see each
%other at all times. This means that the number of messages and
%interactions by necessity grows by the number of clients squared.
Experiments using multiple instances of a single-threaded server are not performed, as having clients distribueted acrosss multiple servers would mean partitioning the clients in areas where they can not interact, making numbers from such a scenario incomparable to the multithreaded solutions.
%
%\subsection{Limitations}

The LEARS approach does have \textit{limitations} and is for example not
suitable if the outcome of a message put restrictions on an object's
state. This is mainly a game design issue, but situations such as
trades can be accommodated by doing full transactions. 
The following
example where two players trade illustrates the problem: Player A
sends a message to player B where he proposes to buy her sword for X
units. After this is sent, player C steals player A's money, and
player A is unable to pay player B should the request go through. 
This is only a problem for trades \textit{within} a single game tick
where the result of a message to another object puts a constraint on
the original sender, and can be solved by means such as putting the
money in escrow until the trade has been resolved, or by doing a
transaction outside of LEARS (such as in a database).
%
Moreover, the design also adds some overhead in that the code is
somewhat more complex, i.e., all communication between elements in the
system needs to go through message queues. The same issue will also
create some runtime overhead, but our results still demonstrate a
significant benefit in terms of the supported number of clients.

Tasks in a thread pool can not be pre-empted, but the threads used for execution can. This distinction creates an interesting look into the performance trade-off of pre-emption. If the number of threads in the threadpool is equal to the number of CPU cores, we have a fully cooperative multitasking system. Increasing the number of threads allow for more pre-emption, but introduces context-switching overhead.
%\PH{Can the approach taken be ALSO be discussed in the context of the
%  keywords given in the CfP ? E.g., How do we use/schedule/utilize the
%  resources of a server compared to existing work?}
