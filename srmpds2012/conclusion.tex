\section{Conclusion}
\label{sec:conclusion} 
In this paper, we have shown that we can improve resource utilization
by distributing load across multiple CPUs in a unified memory
multi-processor system. This distribution is made possible by relaxing
constraints to the ordering and atomicity of events. The system scales
well, even in the case where all players must be aware of all other
players and their actions. The thread pool system balances load well
between the cores, and its queue-based nature means that no task is
starved unless the entire system lacks resources. Message passing
through the blocking queue allows objects to communicate intensively
without blocking each other. Running our prototype game, we show that
the 8-core server can handle twice as many clients before the
response time becomes unacceptable.

\section{Future work}\label{sec:fw}
From the research described in this paper, a series of further
experiments present themselves.
%
The relationship between linearly scaling load and quadratic load can
be tweaked in our implementation. This could answer questions about
which type of load scale better under multi-threaded
implementations. Ideally, the approach presented here should be
implemented in a full, complete massive multiplayer game. This should
give results that are fully realistic, at least with respect to this
specific game.

Another direction this work could be extended is to go beyond the
single shared memory computer used and distribute the workload
across clusters of computers. This could be achieved by implementing
cross-server communication directly in the server code, or by using
existing technology that makes cluster behave like shared memory
machines. 

Furthermore, all experiments described here were run with an update
frequency of 10~Hz. This is good for many types of games, but
different frequencies are relevant for different games. Investigating
the effects of running with a higher or lower frequency of updates on
server performance could yield interesting results.

If, during the implementation of a complex game, it is shown that some
state changes must be atomic to keep the game state consistent, the
message passing nature of this implementation means that we can use
read-write-locks for any required blocking. If such cases are found,
investigating how read-write-locking influence performance would be
worthwhile.
%
%Looking at the scope of garbage-collected languages, investigating how
%different garbage collectors compare for this system would be
%interesting. Investigating the newer multithreaded garbage collectors
%would be particularly interesting. Will they make the garbage
%collection overhead more manageable?  This naturally leads to the
%question of how the system will work without any garbage collection,
%by comparing the current Java implementation with a system written in
%a language with manual memory management, preferably C++.
