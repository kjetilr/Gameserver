\section{Introduction}
Over the last decade, online multi-player gaming has experienced an
amazing growth. Providers of the popular online games must deliver a
reliable service to thousands of concurrent players meeting strict
processing deadlines in order for the players to have an acceptable
quality of experience (QoE). 

%% Game types and server architectures
%% Game server architecture models.
%% -FPS, RTS and small, local servers with low pings.
%% -Separated gameworld instances, each with < 1000 players
%% -Shards, grids and area of interest.

%% Game designs that allow thousands of players to interact in a small
%% space is also challenging.

%#\PH{Note that we are submitting to a scheduling and resource
%  management workshop - which is probably fine - but the abstract,
%  intro and conclusion is targeted towards multimedia communities like
%  NetGames and NOSSDAV. We should look at the CfP
%  (http://www.mcs.anl.gov/~kettimut/srmpds/) and use some of the
%  keywords used here. In other words, we should make some conclusions
%  in all these three sections that say something about the keywords in
%  the CFP!!! Thus, if this could be added (we do have room) where we
%  describe our approach, it would be better for this conference...:
%  ``resource allocation/management/scheduling, load sharing/balancing,
%  performance, ... (these terms should be used ;-) )'', }

One major goal for large game providers is to support as many
concurrent players in a game-world as possible while preserving the
strict latency requirements in order for the players to have an
acceptable quality of experience (QoE). Load distribution in these systems is typically achieved by partitioning game-worlds into areas-of-interest to
minimize message passing between players and to
allow the game-world to be divided between servers. Load balancing is usually completely static, where each area has dedicated hardware. This approach is,
however, limited by the distribution of players in the game-world, and
the problem is that the distribution of players is heavy-tailed with
about 30\% of players in 1\% of the game area~\cite{chen-2006}. To
handle the most popular areas of the game world without reducing the
maximum interaction distance for players, individual spatial
partitions can not be serial. An MMO-server will experience the most CPU load while the players experience the most
``action''. Hence, the worst case scenario for the server is when a
large proportion of the players gather in a small area for high
intensity gameplay.

In such scenarios, the important metric for online multi-player games
is latency. Claypool et. al.~\cite{claypool++-2006} classify different
types of games and conclude that for first person shooter (FPS) and
racing games, the threshold for an acceptable latency is 100ms. For
other classes of networked games, like real-time strategy (RTS) and
massively multi-player online games (MMOGs) players tolerate
somewhat higher delays, but there are still strict latency
requirements in order to provide a good QoE. The accumulated latency
of network transmission, server processing and client processing adds
up to the latencies that the user is experiencing, and reducing any of
these latencies improves the users' experience.

The traditional design of massively multi-player game servers rely on
\textit{sharding} for further load distribution when too many players
visit the same place simultaneously. Sharding involves making
a new copy of an area of a game, where players in different copies are
unable to interact. This approach eliminates most requirements for
communication between the processes running individual shards. An
example of such a design can be found in~\cite{chu-2008}.

The industry is now experimenting with implementations that allow for
a greater level of parallelization. One known example is Eve Online
\cite{drain-2008} where they avoid \textit{sharding} and allow all
players to potentially interact. Large-scale interactions in Eve
Online are handled through an optimized database. On the local scale,
however, the server is not parallel, and performance is extremely
limited when too many players congregate in one area. With LEARS, we
take this approach even further and focus on how many players that can
be handled in a single segment of the game world. We present a model
that allows for better resource utilization of multi-processor, game
server systems which should not replace spatial partitioning
techniques for work distribution, but rather complement them to
improve on their limitations. Furthermore, a real prototype game is
used for evaluation where captured traces are used to generate server
load. We compare multi-threaded and single-threaded implementations in
order to measure the overhead of parallelizing the implementation and
showing the experienced benefits of parallelization. The change in
responsiveness of different implementations with increased load on the
server is studied, and we discuss how generic elements of this game
design impact the performance on our chosen platform of
implementation.

Our results indicate that it is possible to design an ``embarrassingly
parallel'' game server. We also observe that the implementation is
able to handle a quadratic increase of in-server communication when
many players interact in a game-world hotspot. 
%\PH{Thus, here we
%  should add some findings/conclusions/results in the area most
%  relevant to the CFP''}

The rest of the paper is organized as follows: In
section~\ref{sec:concept}, we describe the basic idea of LEARS, before
we present the design and implementation of the prototype in
section~\ref{sec:implementation}. We evaluate our prototype in
section~\ref{sec:eval} and discuss our idea in
section~\ref{sec:disc}. In section~\ref{sec:related}, we put our idea
in the context of other existing work. Finally, we summarize and conclude
the paper in section~\ref{sec:conclusion} and give directions for
further work in section~\ref{sec:fw}.

%\PH{Another question is whether RW could be moved second (after the
%  intro) so that it is better used as a MOTIVATION why we need LEARS?
%  The last paragraph of RW is already highlighting the shortcomings
%  of existing work - if we here really emphasize what is missing it is
%  a good transition to the LEARS idea which would then come
%  next.... (if so the outline paragraph above must also be changed.) -
%  NOTE - this is last priority, first is to include more keywords from
%  cfp}
