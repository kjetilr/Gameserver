\section{LEARS: The Basic Idea}\label{sec:concept}
Traditionally, game servers have been implemented much like game
clients. They are based around a main loop, which updates every active
element in the game. These elements include for example player
characters, non-player characters and projectiles. The simulated world
has a list of all the active elements in the game and typically calls
an ``update'' method on each element. The simulated time is kept
constant throughout each iteration of the loop, so that all elements get updates at the same points in simulated time. This point in
time is referred to as a \textit{tick}.  Using this method, the active
element performs all its actions for the tick. Since only one
element updates at a time, all actions can be performed directly. The
character reads input from the network, performs updates on itself
according to the input, and updates other elements with the results of
its actions.

%
%To make a parallel game server with minimal locking, the system needs
%to be designed from the ground up with parallelism in mind.
LEARS is a game server model with support for lockless, relaxed-atomicity
state-parallel execution.  The main concept is to split the
game server executable into lightweight threads at the finest possible
granularity. Each update of every player character, AI opponent and
projectile runs as an independent work unit. 
%Using this approach, the
%theoretical parallelism is proportional to the load on the
%server \CG{(I don't get it?)}. To do this \CG{(you don't mention before that
%you're doing anything)}, we must relax the presumed deterministic
%requirements of a game server, and we \CG{\sout{will}} show that this approach
%retains consistency and is applicable to real-world games.
%
%\CG{(That sounds too fluffy for me. Shouldn't the argument be somewhat more
%concrete, similar to this:
%The cycles in a highly
%interactive, multiuser game are rather short, so short actually that
%client-server latency differences between players are so large compared
%to the cycle length that the order of events as they are generated by the
%client is irrelevant. Similarly, the order of arrival of these events at
%the server becomes irrelevant because of these differences. If we can
%assume that every cycle can be considered an atomic tick, then state can
%actually be double-buffering between cycles. Processing an event does then
%require reading state from the previous cycle, processing independently
%from all other input events, and writing to the new cycle. If the writing
%destinations are distinct, locking becomes unnecessary.'')}

White et al.~\cite{white++2008} describe a model they call a 
\textit{state-effect pattern}. Based on the observation that changes in a large, actor-based simulation are happening \textit{simultaneously}, they separate read and write operations. Read operations work on a consistent previous state, and all write operations are batched and executed to produce the state for the next tick. This means that the ordering of events scheduled to execute at a tick does not need to be considered or enforced. 
For the design in this paper, we additionally remove the requirement for batching of write operations, allowing these to happen anytime during the tick. The rationale for this relaxation is found in the way traditional game servers work. In the traditional single-threaded main-loop approach, every update is allowed to change any part of the simulation state at any time. In such a scenario the state at a given time is a combination of values from two different points in time, current and previous, exactly the same situation that occurs in the design presented here.
%
%This is the case for many
%games and is mainly an issue of game design, i.e., what is the desired
%behavior if two players perform conflicting
%actions at the same instant. In the
%traditional main-loop approach, every event in a
%game scheduled for a tick are executed by the main loop. The
%main loop process these in arrival order. Thus, the ordering is highly
%influenced by the client latencies and at which point in time between
%two ticks the event was dispatched by the
%client. Remember that a tick is the smallest amount of time
%considered by the game, and this means there is no correct order to
%execute conflicting events within a single tick. As such, the
%ordering of events scheduled for a tick in a traditional main
%loop is \textit{not} deterministic. LEARS takes advantage of this
%relaxation and allows events scheduled
%for a tick to execute in any order.

The second relaxation relates to the atomicity of game state updates. The fine
granularity creates a need for significant communication between threads
to avoid problematic lock contentions. Systems where
%
elements can only update their own state and read any state
without locking~\cite{Abdelkhalek2004++}
%
do obviously not work in all cases. However, game servers are not
accurate simulators, and again, depending on the game design, some
(internal) errors are acceptable without violating game state
consistency. 
Consider
the following example: Character A moves while character B attacks. If
only the X coordinate of character A is updated at the point in time
when the attack is executed, the attack sees character A at a position with the new X coordinate and the old Y coordinate.  This position is within the accuracy of the
simulation which in any case is no better than the distance an object
can move within one tick. 

On the other hand, for actions where a margin of error is not
acceptable, transactions can be used keeping the object's state
internally consistent. However, locking the state is
expensive. Fortunately, most common game actions do not require
transactions, an observation that we take advantage of in LEARS.
%but if two variables in an object's game state must be
%altered simultaneously to retain consistency, locking must be used.

These two relaxations allow actions to be performed on game objects in
any order without global locking. It can be implemented using message
passing between threads and retains consistency for most game actions.
This includes actions such as moving, shooting, spells and so forth.
Consider player A shooting at player B: A subtracts her ammunition state, and
send bullets in B's general direction by spawning bullet objects. The
bullet objects runs as independent work units, and if one of them hits player B, it
sends a message to player B. When reading this message, player B
subtracts his health and sends a message to player A if it reaches zero.
Player A then updates her statistics when she receives player B's message. This series of events can be time critical at certain points. The most important point is where the decision is made if the bullet hits player B. If player B is moving, the order of updates can be critical in deciding if the bullet hits or misses. In the case where the bullet moves first, the player does not get a chance to move out of the way. This inconsistency is however not a product of the LEARS approach. Game servers in general insert active items into their loops in an arbitrary fashion, and there is no rule to state which order is ``correct''.

The end result of our proposed design philosophy is that there is no
synchronization in the server under normal running conditions. Since
there are cases where transactions are required, they can be implemented
outside the LEARS event handler running as transactions
requiring locking. In the rest of the paper, we consider a
practical implementation of LEARS, and evaluate its performance and
scalability.
