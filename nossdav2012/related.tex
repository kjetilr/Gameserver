\section{Related Work}\label{sec:related}
At Netgames 2011~\cite{raaen++2011}, we presented a demo with a preliminary version of LEARS.
%
There is also some earlier research on how to optimize game server
architectures for online games, both MMOGs and smaller-scale games. In
this section, we summarize some of the most important findings from
related research in this field. 
%
For example, "Red Dwarf", the community-based successor to "Project
Darkstar" by Sun Microsystems~\cite{waldo-2008}, is a good example 
of a parallel approach to game server design. Here, response time is 
considered one of the most important metrics for game server performance, 
and suggests a parallel approach for scaling. The described system 
uses transactions for all updates to world state, including player 
position. This differs from LEARS, which invastigates the case for common actions where atomicity of transactions is not necessary. 

Work has also been done on scaling games by looking at the
optimization as a data management problem. The authors
in~\cite{white-2007} have developed a highly expressive scripting
language called SGL that provide game developers a data-driven AI
scheme for non-player characters. By using query processing and
indexing techniques, they can efficiently scale to a large number of
non-player objects in games.
%
Moreover, Cai et al.~\cite{Cai2002++} present a scalable architecture
for supporting large-scale interactive Internet games. Their approach
divide the game world into multiple partitions and assign each
partition to a server. The issues with this solution is that the
architecture of the game server is still a limiting factor in worst
case scenarios as only a limited number of players can interact in the
same server partition at a given time. 
%
There have also been proposed
several middleware systems for automatically distributing the game
state among several participants. In~\cite{Glinka2007++}, the authors
present a middleware which allows game developers to create
large, seamless virtual worlds and to migrate zones between
servers. This approach does, however, not solve the challenge of many
players that want to interact in a popular
area. 
The research presented in~\cite{Muller2007++} shows that proxy
servers are needed to scale the number of players in the game, while the authors discuss the possibility of
using grids as servers for MMOGs. 

%% Beskow et al.~\cite{Beskow2009++} have also been
%% investigating partitioning and migration of game servers. Their
%% approach uses core selection algorithms to locate the most optimal
%% server.
%% Low latency in the network is important to be able to support the
%% deadlines for online games. In~\cite{Petlund2009}, games and
%% interactive applications using TCP was shown to receive reduced
%% retransmission latency when the retransmission mechanisms was modified
%% to 
%% We have worked on how to reduce latency by
%% modifying the TCP protocol to better support time-dependent
%% applications~\cite{Petlund2009}.  However, the latency is not only
%% determined by the network, but also the response time for the game
%% servers. If the servers have a to large workload, the latency will
%% suffer.

In~\cite{Abdelkhalek2003++}, the authors are discussing the behavior
and performance of multi-player game servers. They find that in the
terms of benchmarking methodology, game servers are very different
from other scientific workloads. Most of the sequentially implemented
game servers can only support a limited numbers of players, and the
bottlenecks in the servers are both game-related and
network-related. The authors in ~\cite{Abdelkhalek2004++} extend their
work and use the computer game Quake to study the behavior of the game. When
running on a server with up to eight processing cores the game suffers
because of lock synchronization during request processing. High wait
times due to workload imbalances at global synchronization points are
also a challenge.

There have been a lot of research has been performed on
how to partition the server and scale the number of players by
offloading to several servers.  Modern game servers have also been
parallelized to scale with more processors. However, a large amount of
processing time is still wasted on lock synchronization. In our game
server design, we provide a complementary solution and try to
eliminate the global synchronization points and locks, i.e., making the
game server ``embarrassingly parallel'' which aims at
increasing the number of concurrent users per machine.

