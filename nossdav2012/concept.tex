\section{LEARS: The Basic Idea}\label{sec:concept}
Traditionally, game servers have been implemented much like game
clients. They are based around a main loop, which updates every active
element in the game. These elements include for example player
characters, non-player characters and projectiles. The simulated world
has a list of all the active elements in the game and typically calls
an ``update'' method on each element. The simulated time is kept
constant throughout each iteration of the loop, so that all elements
will get updates at the same points in simulated time. This point in
time is refered to as a \textit{tick}.  Using this method, the active
element will perform all its actions for the tick. Since only one
element updates at a time, all actions can be performed directly. The
character reads input from the network, performs updates on itself
according to the input, and updates other elements with the results of
its actions.

%
%To make a parallel game server with minimal locking, the system needs
%to be designed from the ground up with parallelism in mind.
LEARS is a game server model with support for lockless, relaxed
atomicity state parallel execution.  The main concept is to split the
game server executable into lightweight threads at the finest possible
granularity. Each update of every player character, AI opponent and
projectile runs as an independent work unit. Using this approach, the
theoretical parallelism will be proportional to the load on the
server. To do this, we must relax the presumed deterministic
requirements of a game server, and we will show that this approach
retains consistency and is applicable to real-world games.

Our claim is that the ordering of events scheduled to execute at a
tick does not always need to be considered. This is the case for many
games and is mainly an issue of game design, i.e., what is the desired
behavior if two players perform conflicting
actions at the same instant. In the
traditional main-loop approach, every event in a
game scheduled for a tick are executed by the main loop. The
main loop process these in arrival order. Thus, the ordering is highly
influenced by the client latencies and at which point in time between
two ticks the event was dispatched by the
client. Remember that a tick is the smallest amount of time
considered by the game, and this means there is no correct order to
execute conflicting events within a single tick. As such, the
ordering of events scheduled for a tick in a traditional main
loop is \textit{not} deterministic. LEARS takes advantage of this
relaxation and allows events scheduled
for a tick to execute in any order.

The second relaxation relates to game state consistency. The fine
granularity creates a need for significant communication between threads
to avoid problematic lock contentions. Systems where
%
elements can only update their own state and read any state
without locking~\cite{Abdelkhalek2004++}
%
will obviously not work in all cases. However, game servers are not
accurate simulators, and again, depending on the game design, some
(internal) errors are acceptable without violating game state
consistency. 
%Consider
%the following example: Character A moves while character B attacks. If
%only the X coordinate of character A is updated at the point in time
%when the attack is executed, the attack will see character A at position
%$(X_{t+\Delta T},Y_{t})$.  This position is within the accuracy of the
%simulation which in any case is no better than the distance an object
%can move within $\Delta T$. The only requirement for this to work is
%that assignment operations are atomic. 
%
On the other hand, for actions where a margin of error is not
acceptable, transactions can be used keeping the object's state
internally consistent. However, locking the state is
expensive. Fortunately, most common game actions do not require
transactions, an observation that we take advantage of in LEARS.
%but if two variables in an object's game state must be
%altered simultaneously to retain consistency, locking must be used.

These two relaxations allow actions to be performed on game objects in
any order without global locking. It can be implemented using message
passing between threads and retains consistency for most game actions.
This includes actions such as moving, shooting, spells and so forth.
%Consider player A shooting at player B: A subtracts her ammunition state, and
%send bullets in B's general direction by spawning bullet objects. The
%bullet objects runs as independent threads, and if one of them hits player B, it
%will send a message to player B. When reading this message, player B
%subtracts his health and sends a message to player A if it reaches zero.
%Player A then updates her statistics when she receives player B's message.
The end result of our proposed design philosophy is that there is no
synchronization in the server under normal running conditions. Since
there are cases where transactions are required, they can be implemented
outside the LEARS event handler running as transactions
requiring locking. In the rest of the paper, we will consider a
practical implementation of LEARS, and evaluate its performance and
scalability.
